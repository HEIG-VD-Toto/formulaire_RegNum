
\hformbar
\formdesc{Lieu des poles(Root locus)}

\footnotesize Outil important permettant la synthèse de systèmes réglés : 

pôles complexes conjugués :

{\hfill $s_{1,2} = -\delta \pm j\omega $\hfill}

Réponse indicielle :

{\hfill $g(t) = e^{-\delta \cdot t} \cdot sin(\omega t) $\hfill}

Rappel (formulaire régulation automatique) : 

{\hfill $T_{reg} = \cfrac{3}{\delta} = \cfrac{-3}{\mathbb{R}(s_f)}$\hfill}\vspace{3mm}

Règles de tracer du lieu des pôles (L.d.P) :
\begin{enumerate}
    \item L.d.P a n branches = degré du dénominateur 
    \item L.d.P a M branches = degré du numérateur
    \item L.d.P symétrique par l'axe $\mathbb{R}$
    \item Points de départ à $K_p = 0$
    \item Points de fin à $K_p = \infty$
    \item $d = n-m$ branches restante partent en asymptote infinie, où d correspond au degré relatif de la B.O\\ les asymptotes forment des étoiles régulières
    \item Tout point de l'axe réel situé à gauche d'un nombre impair de pôles et de zéros réels fait partie du lieu
\end{enumerate}

\tikzset{cross/.style={cross out, draw=black, minimum size=2*(#1-\pgflinewidth), inner sep=0pt, outer sep=0pt},
cross/.default={4pt}}

\tikzset{ % this style creates an arrow like the one you draw in the middle of a path
   ->-/.style={decoration={markings,mark=at position 0.5 with {\arrow{Straight Barb}}},
               postaction={decorate}}
}
%default radius will be 1pt. 
\begin{tikzpicture}[]
    \tikzmath{
    coordinate \x;
    \x1=(-2,0);
    \x2=(0,0);
    \x3=(-2.5,0);
    \x4 = (-3,0);
    \x5 = (-1,0);
    }
    
    \draw[help lines,dashed] (-5.5,-1.5) grid (0.5,1.5);
    \draw[->] (-6,0) -- (1,0) node [anchor=west] {$R$};
    \draw[->] (0,-1.5) -- (0,1.5) node [anchor=south] {$Img$};
    \draw[->-,red,thick] (\x1) -- (\x5);
    \draw[->-,red,thick] (\x5) arc (0:180:1);
    \draw[->-,red,thick] (\x4) -- (\x3);
    
    \draw[->-,blue,thick] (\x2) -- (\x5);
    \draw[->-,blue,thick] (\x5) arc (0:-180:1);
    \draw[->-,blue,thick] (\x4) -- (-6,0);
    \draw (\x1) node[cross] {};
    \draw (\x2) node[cross] {};
    \draw (\x3) circle (4pt);
    
\end{tikzpicture}


\hformbar

